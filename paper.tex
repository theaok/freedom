%to have line numbers
%\RequirePackage{lineno}
\documentclass[10pt, letterpaper]{article}      
\usepackage[margin=.1cm,font=small,labelfont=bf]{caption}[2007/03/09]
%\usepackage{endnotes}
\usepackage{setspace}
\usepackage{longtable}                        
\usepackage{anysize}                          
%\bibpunct{(}{)}{,}{a}{,}{,}                   
%\bibpunct{(}{)}{,}{a}{}{,}                   
\usepackage{amsmath}
\usepackage[pdftex]{graphicx}  %[pdftex]git latex doesn't like it             
\usepackage{epstopdf}
\usepackage{hyperref}                             % For creating hyperlinks in cross references


% \usepackage[margins]{trackchanges}

% \note[editor]{The note}
% \annote[editor]{Text to annotate}{The note}
%    \add[editor]{Text to add}
% \remove[editor]{Text to remove}
% \change[editor]{Text to remove}{Text to add}



\marginsize{1cm}{1cm}{.5cm}{.5cm}%{left}{right}{top}{bottom}   
					          % Helps LaTeX put figures where YOU want
 \renewcommand{\topfraction}{1}	                  % 90% of page top can be a float
 \renewcommand{\bottomfraction}{1}	          % 90% of page bottom can be a float
 \renewcommand{\textfraction}{0.0}	          % only 10% of page must to be text

 \usepackage{float}                               %latex will not complain to include float after float

\usepackage[table]{xcolor}                        %for table shading
\definecolor{gray90}{gray}{0.90}
\definecolor{orange}{RGB}{255,128,0}

\renewcommand\arraystretch{.9}                    %for spacing of arrays like tabular

\newenvironment{ig}[1]{
\begin{center}
 %\includegraphics[height=5.0in]{#1} 
 \includegraphics[height=3.3in]{#1}
\end{center}}

 \newcommand{\cc}[1]{
\hspace{-.13in}$\bullet$\marginpar{\begin{spacing}{.6}\begin{footnotesize}{#1}\end{footnotesize}\end{spacing}}
\hspace{-.13in} }

\usepackage{datetime}


%\usepackage[latin1]{inputenc} %git latex compiler doesn't likeit
\usepackage{tikz}
\usetikzlibrary{shapes,arrows,backgrounds}


%\usepackage{color}					% For creating coloured text and background
%\usepackage{float}
\usepackage{subfig}                                     % for combined figures

\renewcommand{\ss}[1]{{\colorbox{blue}{\bf \color{white}{#1}}}}
\newcommand{\ee}[1]{\endnote{\vspace{-.10in}\begin{spacing}{1.0}{\normalsize #1}\end{spacing}\vspace{.20in}}}




\usepackage{sectsty}
\allsectionsfont{\normalfont\sffamily}



\usepackage{sectsty}
\allsectionsfont{\normalfont\sffamily}
%\usepackage[margins]{trackchanges}

\renewcommand\familydefault{\sfdefault}

\usepackage{verbatim}
\usepackage{rotating}
\usepackage{catchfilebetweentags}
%-------------------- END extra options -----------------------------------------
\date{Draft: {}\today}
\title{title\footnote{Author Contributions:
  XXX designed
  research. XXX performed research. XXX analyzed data. XXX wrote the paper.
% author contribution statements. See: http://www.nature.com/news/publishing-credit-where-credit-is-due-1.15033
}
}
\author{Adam Okulicz-Kozaryn
\thanks{EMAIL: adam.okulicz.kozaryn@gmail.com
\hfill  I thank XXX.  All mistakes are mine.} 
}
% {\small Rutgers, The State University of New Jersey, Camden}\\
% Micah Altman\thanks{EMAIL: micah.altman@gmail.com
%   \hfill } \\
% {\small Massachusetts Institute of Technology}


\begin{document}

%%\setpagewiselinenumbers
%\modulolinenumbers[1]
%\linenumbers

%\bibliographystyle{/home/aok/papers/root/tex/pnas2011.bst}

\maketitle
\vspace{-.4in}
\begin{center}
\end{center}


\vspace{.15in} 
\noindent{\sc keywords:  }
%\vspace{-.25in} 

\begin{abstract}
\noindent  
\end{abstract}
\begin{spacing}{1.0}


%\bibliography{/home/aok/papers/root/tex/ebib.bib}
\begin{thebibliography}{10}

\bibitem{mackay08}
MacKay D (2008) {\em Sustainable Energy-without the hot air}.
\newblock (UIT Cambridge).
\end{thebibliography}

\section{Data}

Budget balance/GDP is a sufficient indicator on the sustainability of public finance and I think should be included in our set of covariates. For the inflation rate, I suggest to derive two measures directly capturing price stability, (i) inflation rate per se, and (ii) standard deviation of inflation rate where the latter adequetly captures the aggregate price uncertainty which of course is a drawback on market transactions. I also looked at the literature a bit for novel but well-established measures of economic freedom. I think we should build a clear argument that economic freedom itself is a summary of various institutional components primarily centered around (i) security of physical and intellectual property rights, (ii) transaction costs, (iii) fiscal institutions, and (iv) social institutions. This is quite clearly a non-trivial detour from the classical conceptual framework on economic freedom eschewed by Hayek, Friedman and the Austrian school and an approach resembling the original idea by Isaac Berlin on the positive economic freedom.

One measure which I think we should not hesitate to construct is contract-intensive money which had been introduced by Clauge-Keefer team back in 1999 (WP version of the paper is avalible here https://ideas.repec.org/p/pra/mprapa/25717.html while the paper has been published by JEG) and recently by Prados de la Escosura (\url{http://orff.uc3m.es/bitstream/handle/10016/4677/contract_padros_C_2009.pdf?sequence=1}) building an explanation for Argentina's long-run decline. This is a composite measure of the ease of contract enforcement and security of property rights as is defined as follows: (M2 money supply - currency outside the banks)/(M2 money supply). Large sums of currency outside the banks simply indicates insecure property rights and high costs of contract enforcement which is nothing else than a mere reflection of low economic freedom. For instance, Argentina had a high CIM rate until 1920s during the Belle Epoque but experience a continuous decline ever since. I did not come across a study using CIM for a large number of countries. Using the data at IMF and World Bank would suffice the construction of such an indicator

\newpage
\section*{\huge ONLINE SUPPLEMENTARY MATERIAL}



\end{spacing}
\end{document}
